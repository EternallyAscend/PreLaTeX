\section[进一步使用]{进一步使用\LaTeX \, Beamer}

\begin{frame}
    \frametitle{章节拆分}
    \begin{block}{引入拆分的章节文件}
        \textbackslash include\{sections/background\}\\
        \textbackslash include\{sections/basic\}\\
        \textbackslash include\{sections/advance\}\\
        \textbackslash include\{sections/summary\}\\
    \end{block}
    可以将不同章节拆分后编写。当论文篇幅较长时,更容易查找文章对应章节的内容。
\end{frame}

\begin{frame}
    \frametitle{参考文献}
    GitHub上的一个\LaTeX \, Beamer示例\footfullcite{gitHubRepo}。
    \begin{block}{reference.bib文件}
        \small
        @misc\{gitHubRepo,\\
        \quad title=\{PreLaTeX\},\\
        \quad author=\{EternallyAscend\},\\
        \quad howpublished=\{https://github.com/EternallyAscend/PreLaTeX\}\\
        \}
    \end{block}
    编写好reference.bib后,需要在tex主文件中引入。\\
    同时需要注意biber编译过程。
\end{frame}

\begin{frame}
    \frametitle{参考文献}
    \begin{block}{tex主文件,此处仅引入宏包和reference.bib,未设置格式}
        \small
        \textbackslash usepackage[backend=biber,sorting=none]\{biblatex\}\\
        \textbackslash addbibresource\{reference.bib\}
    \end{block}
    \begin{block}{文内引用}
        \small
        GitHub上的一个\textbackslash LaTeX \textbackslash , Beamer示例 \textbackslash footfullcite\{gitHubRepo\}。
    \end{block}
\end{frame}

\begin{frame}
    \frametitle{推荐编译方法}
    \begin{block}{效果预览}
        直接使用xelatex main命令即可
    \end{block}
    \begin{block}{编译成文}
        由于涉及目录和参考文献。按照xelatex -> biber -> xelatex -> xelatex的流程编译。
    \end{block}
    .vscode文件夹内设置了VSCode中的编译命令。
\end{frame}

\begin{frame}
    \frametitle{动画}
    \begin{columns}
        \begin{column}{0.7\textwidth}
            \begin{block}{无序列表}<1->
                \textbackslash begin\{block\}\{基础Block\}<1->\\
                    \quad \textbackslash dots\\
                \textbackslash end\{block\}\\
                \textbackslash begin\{exampleblock\}\{举例Block\}<2->\\
                    \quad \textbackslash dots\\
                \textbackslash end\{exampleblock\}\\
                \textbackslash begin\{alertblock\}\{警告Block\}<3->\\
                    \quad \textbackslash dots\\
                \textbackslash end\{alertblock\}
            \end{block}
        \end{column}
        \begin{column}{0.2\textwidth}
            \begin{block}{基础Block}<1->
                \dots
            \end{block}
            \begin{exampleblock}{举例Block}<2->
                \dots
            \end{exampleblock}
            \begin{alertblock}{警告Block}<3->
                \dots
            \end{alertblock}
        \end{column}
        \begin{column}{0.10\textwidth}
        \end{column}
    \end{columns}
\end{frame}
